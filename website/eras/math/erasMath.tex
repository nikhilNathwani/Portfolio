\documentclass[a4paper,11pt]{article}
\usepackage{amsthm}
\usepackage{amsmath}
\usepackage{amssymb}
\usepackage{color}
\usepackage{graphicx}
\usepackage{fullpage}
\usepackage{hyperref}
\usepackage{verbatim}
\usepackage{tikz}
\usepackage[ruled]{algorithm2e}
\usepackage{fontenc}
\usepackage[utf8]{inputenc}
\usepackage{mathtools}
\usetikzlibrary{arrows}
\newtheorem{theorem}{Theorem}
\newtheorem{lemma}{Lemma}[section]
\newtheorem*{remark}{Remark}
\newtheorem{proposition}[theorem]{Proposition}
\renewcommand{\thefootnote}{\fnsymbol{footnote}}
\newcommand{\norm}[1]{\lVert#1\rVert}
       % define the title
       \author{Nikhil J. Nathwani}
       \title{Eras of the NBA}
       \begin{document}
       % generates the title
       \maketitle

\section{Goal}
The goal is to delineate the distinct eras of each NBA franchise. Taking a player-centric approach, this is equivalent to determining which players define the history of each franchise. I've chosen to quantify a player's worth based on his win share\footnote{A measure that tries to quantify how many regular-season wins an individual accounted for. Complete explanation \textcolor{blue}{\underline{\href{http://www.basketball-reference.com/about/ws.html}{here}}}.} percentage\footnote{Simply calculated by dividing an individual's win shares by the sum of the win shares of all members of his team.} (WS\%), because a player with a large percentage of his team's win shares in a given season is likely to be crucial to that era of the franchise. Other metrics could be used to gain different perspectives on a team's history.

The method used to select these ``franchise-definers'' should keep the number of chosen players to a minimum, in order to weed out those who were not true flagships of any particular era. But of course, the obvious superstar players should not be excluded either.

\section{Method}
Consider an arbitrary team $X$. We would like to determine the set of franchise-defining players for $X$. Let:\\

$P_j=$ the set of players on team $X$ in season $j$.\\ 


$w_j(p) = \left\{ \begin{array}{ll}
		            \text{WS\% of player } p \text{ in season } j & \quad p \in P_j\\
		            0 & \quad \text{otherwise}
		        \end{array}
 		   \right. $\\



As a preprocessing step, I use the following algorithm to exclude a number of players that clearly did not have a high enough impact:\\

	\begin{algorithm} [H]
	\caption{\texttt{Preprocess}($T$)}
	\For{each season $j$} {
		$P^* := \emptyset$\;
		$totalWS := 0$\;
       	\While{$totalWS < T$} {
		$p := \arg \! \max \limits_{p \in P_j} \{w_j(p)\}$\;
		$totalWS \leftarrow totalWS + w_j(p)$\;
		$P_j \leftarrow P_j \setminus \{p\}$\;
		$P^* \leftarrow P^* \cup \{p\}$\;
	}
	$P_j \leftarrow P^*$\;
	}
	\end{algorithm}


\textcolor{white}{spacing}

In words, this retains the win share leaders of team $X$ each season, but only keeps as many as are needed to cover $T\%$ of the total win shares.

To make the problem more tractable and interesting, I impose the following constraint:

\begin{equation}
\forall_j  \; \sum_{p \in P} w_j(p) \ge t
\end{equation}

\noindent where $t$ is a tweakable parameter\footnote{See the ``Table of Parameters'' in the ``Appendix'' section for a table showing the parameters $T$ (from \texttt{Preprocess}) and $t$ chosen for each team.}. Ensuring that $t\%$ of the total win shares are accounted for each season allows every season to be well-represented in team $X$'s history.

What we're now left with is a \textcolor{blue}{\underline{\href{http://en.wikipedia.org/wiki/Covering_problems}{covering problem}}}: we want to satisfy the constraints specifed by inequality (1) while minimizing the total number of players used. Since were dealing with decision variables (i.e. 0,1-valued variables that specify whether or not a player is used), we can formulate this problem as an \textcolor{blue}{\underline{\href{http://en.wikipedia.org/wiki/Integer_programming}{integer program}}} (IP). But before diving into the IP formulation, I want to define the decision variables in a way that is more sensible for this real-world scenario:

If we select player $p$ in season $j$, and player $p$ is also available to be selected in season $j+1$, then it is befitting to select player $p$ in season $j+1$. This is because we know player $p$ was valuable to the team in both of those years (because of the \texttt{Preprocess} algorithm), and being valuable for multiple consecutive seasons certainly warrants inclusion in team $X$'s historical record.

So, let's define a \emph{reign} to be a maximal-length sequence of consecutive seasons in which a specific player is in the preprocessed seasonal player sets. E.g. if player $p$ is in $P_0,P_1,P_2,P_4,$ and $P_5$, then $\{0,1,2\}$ and $\{4,5\}$ are reigns. Now instead of selecting players, we'll select reigns, thereby selecting the player corresponding to that reign for each season of that reign's duration. 



\section{IP-Formulation}
I'll index the reigns numerically\footnote{In the implementation there is a mapping from reigns to the players they correspond to, but this is irrelevant when discussing the mathematical reasoning.} so that our decision variables are: 

\begin{center}
$r_{i} = \left\{ \begin{array}{ll}
		            1 & \quad \text{reign } i \text{ is selected}\\
		            0 & \quad \text{otherwise}
		        \end{array}
 		   \right. $
\end{center}


Let $W$ be a matrix with entries defined as follows:

\begin{center}
$W_{ij} = \left\{ \begin{array}{ll}
		            \text{WS\% of player $p$ corresp. to reign } i \text{ during season } j & \quad p \in P_j\\
		            0 & \quad \text{otherwise}
		        \end{array}
 		   \right. $
\end{center}

Our problem has been reduced to the following binary integer program, where $\vec{r}$ is the vector of decision variables $r_i$ and $\vec{t}$ a column vector (with length equal to the number of seasons) where each entry is $t$ from inequality (1): 

\begin{center}
    min. $\norm{\vec{r}}_1$\\

    \, \, \, s.t. $W\vec{r} \ge \vec{t}$
    
   \, \, \, $ r_i \in \{0,1\}$ \! $\forall_i$
\end{center}

This IP can be solved using branch-and-cut methods, for which there is a handy library called \textcolor{blue}{\underline{\href{http://www.gnu.org/software/glpk/}{GLPK}}}. The solution to this IP is a binary vector denoting which reigns have been selected. Mapping the selected reigns back to the players the correspond to gives us our desired set of players to represent team $X$'s history.

\section{Appendix}
\subsection{Table of Parameters}
\begin{table}[ht]
  \begin{minipage}[b]{0.6\linewidth}\centering
    \begin{tabular}{|c|c|c|}
    \hline
    Team & $T$ & $t$ \\ \hline
    ATL  & 0.6   & 0.4   \\ \hline
    BOS  & 0.7   & 0.4   \\ \hline
    BRK  & 0.55  & 0.4   \\ \hline
    CHA  & 0.7   & 0.4   \\ \hline
    CHI  & 0.7   & 0.4   \\ \hline
    CLE  & 0.6   & 0.4   \\ \hline
    DAL  & 0.8   & 0.4   \\ \hline
    DEN  & 0.65  & 0.4   \\ \hline
    DET  & 0.6   & 0.4   \\ \hline
    GSW  & 0.6   & 0.4   \\ \hline
    HOU  & 0.7   & 0.4   \\ \hline
    IND  & 0.7   & 0.4   \\ \hline
    LAC  & 0.75  & 0.4   \\ \hline
    LAL  & 0.8   & 0.4   \\ \hline
    MEM  & 0.8   & 0.4   \\ \hline
    \end{tabular}
  \end{minipage}
  \hspace{-.2\linewidth}
  \begin{minipage}[b]{0.6\linewidth}\centering
    \begin{tabular}{|c|c|c|}
    \hline
    Team & $T$ & $t$ \\ \hline
    MIA  & 0.65  & 0.4   \\ \hline
    MIL  & 0.6   & 0.4   \\ \hline
    MIN  & 0.8   & 0.4   \\ \hline
    NOP  & 0.8   & 0.4   \\ \hline
    NYK  & 0.55  & 0.4   \\ \hline
    OKC  & 0.7   & 0.4   \\ \hline
    ORL  & 0.6   & 0.4   \\ \hline
    PHI  & 0.7   & 0.4   \\ \hline
    PHO  & 0.6   & 0.4   \\ \hline
    POR  & 0.6   & 0.4   \\ \hline
    SAC  & 0.7   & 0.4   \\ \hline
    SAS  & 0.6   & 0.4   \\ \hline
    TOR  & 0.7   & 0.4   \\ \hline
    UTA  & 0.6   & 0.4   \\ \hline
    WAS  & 0.55  & 0.4   \\ \hline
    \end{tabular}
  \end{minipage}
\end{table}
\end{document}